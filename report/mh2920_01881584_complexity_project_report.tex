% ProjectTemplate.tex LaTeX template for Complexity and Networks projects
% T.S.Evans, K. Christensen Imperial College, London

\documentclass[a4paper,12pt]{article}

% These packages are standard and very useful for many mathematical symbols but you may well not need them
%\usepackage{amsmath,amssymb,amscd}

% *** FIGURES
% Info on figures at http://en.wikibooks.org/wiki/LaTeX/Floats,_Figures_and_Captions
%
% --- graphicx
\usepackage{graphicx}
\usepackage{algorithmic}
% This is the main one used to read in images.  ESSENTIAL
%
% I DO NOT recommend you try anything else.  Messing around with inages in LaTeX is a pain.
% Try making an image or figure 5\% smaller in LaTeX can make all the difference



% for sub figures
\usepackage{caption}
\usepackage{subcaption}



% Change Page Size
\usepackage[a4paper,margin=2cm, marginparwidth=2cm]{geometry}



% *** HYPERLINKS
%
%\usepackage{hyperref}
%
% See http://en.wikibooks.org/wiki/LaTeX/Hyperlinks
% or hyperref manual.  Main LaTeX reference systems automatically made into hyperlinks in pdf document.
% Again this should always be the last package to be defined which may cause issues with showkeys.
% An out-of-memory error might be caused by this package if its in the wrong position.
% For additional hypelinks in the text try
%   \hyperref[mainlemma]{lemma \ref*{mainlemma}}
%   \url{<my_url>}
%   \href{<my_url>}{<description>}
% Now define replacements which will work even if hyperref not included
\providecommand{\href}[2]{\texttt{#2}}
\providecommand{\url}[1]{\texttt{#1}}


% *** KEYS SHOWN IN DRAFTS
%
%\usepackage{showkeys}
%
% This is great for drafts as it will show all the labels you have defined.
% DON'T forget to comment it out before submission
% Always the last package to define.
% An out-of-memory error might be caused by this package if its in the wrong position.


% This will deal with the headers
%\pagestyle{headings}
%\markboth{CID \texttt{012345678} \hfill Project \hfill }{CID \texttt{012345678} \hfill Project \hfill }


\begin{document}

% Maybe it is just as easy to lay the title page out yourself.
% The following lines are one way to do this.

\begin{center}
 {\Large\textbf{Networks Project}}  \\[3pt]
 {\Large\textbf{A Growing Network Model}} \\[6pt]
 {\large CID: \texttt{01881584}} \\[3pt]
 19th February 2023
\end{center}

% I prefer to lay out the first page myself
%\maketitle


\vspace*{2cm}
\noindent
{\bf Abstract:} You may want to write a concise abstract that briefly puts the
work into context, explains what was done, how it was done, the main results,
the conclusions you could draw and the implications hereof.

\vspace*{2cm}
\noindent
{\bf Word count:}
282 words in report (excluding front page, figure captions, table captions, acknowledgement and bibliography).


\newpage


\section{Implementation of the Oslo model.}

\subsection{TASK 1}

First, I created a class \textit{oslo\_lattice}, which implements the boundary-driven Oslo model. The algorithm itself is fairly simple, mirroring the steps outlined in the project notes. However, one non-trivial thing was the algorithm that deals specifically with relaxation, as it must ensure there are no supercritical sites before driving the system with another grain.

The obvious, \textit{naive} algorithm, would be to keep iterating through all $L$ sites checking for supercriticality, and if a full iteration occurs without encountering a simple supercritical site, relaxation terminates. However, I have devised an optimized algorithm, which I named the \textit{ceilidh algorithm}, which works like this:

\begin{algorithmic}

\STATE $currentIndex\gets 0$
\STATE $maxIndexAffected\gets 0$
\WHILE{$currentIndex < L$ AND $currentIndex \leq maxIndexAffected$}
	\IF {site at $currentIndex$ is supercritical}
		\STATE relax current site
		\STATE $currentIndex\gets currentIndex - 1$
		\STATE $maxIndexAffected\gets \max(maxIndexAffected, currentIndex+1)$
	\ELSE
		\STATE $currentIndex\gets currentIndex + 1$
	\ENDIF
\ENDWHILE

Instead of repeated full iteration, the algorithm keeps making single ``steps forward" and ``steps back", since the relaxation of a site at index $i$ can only cause supercriticalities on sites with indices ranging from $i-1$ to $i+1$, that is, on itself and its direct neighbours. The $maxIndexAffected$ variable applies the same principle to prevent the algorithm from needlessly iterating through the entire lattice when the last supercriticality gets resolved.

\end{algorithmic}

Tests:
\begin{itemize}
\item \textbf{Matching seed should yield deterministic behaviour.} I initialized several instances of the lattice with the same exact initial properties and hardcoded the seed for the random.py package to reset to a constant value before simulating every instance, and ran the simulation with 100 steps on all of them. I expected the states of all instances to be the same. This turned out to be true: all states across all instances were the same after the same amount of steps when resetting the seed each time.\\
This assures that the stochastic behaviour of the programme is effected entiredly by the random.py package, usage of which can be easily monitored.

\item \textbf{For any state, $z_i\leq z^{\mathrm{th}}_i$ always.} I created 10 instances with $L=100$ and simulated 10-100 steps on each. Then I checked whether $z_i\leq z^{\mathrm{th}}_i$ for all $i$ for their states. If this weren't true, it would mean there are supercritical sites not being relaxed before starting the next step.\\
The test concluded well: there were no violations of this principle detected. This assures the relaxation algorithm used is valid.

\item \textbf{The average pile heights match the values in the project notes.} After running the simulation on 10 instances and averaging the values after 500 and 5000 steps respectively, the average value of $h_1$ was found to be $26.7$ for $L=16$ and $53.5$ for $L=32$. These values match the ones given.

\item \textbf{Performance of the relaxation algorithm used shouldn't be worse than the naive algorithm.} In fact, since each found supercriticality increases the number of sites to be checked by $L$ for the naive algorithm and by $2$ for the ceilidh algorithm, the time complexity of ceilidh should be less by one power of $L$ than that of the naive algorithm. Indeed, running a benchmark test confirms this: the ceilidh algorithm always outperforms the naive algorithm, and the ratio of runtimes increases as $L$ increases. TODO INSERT PLOT.
\end{itemize}

\section{The height of the pile $h(t; L)$}
\subsection{TASK 2a}

The set of all possible configurations is a subset of the set of \textbf{all} configurations for which
$$0\leq z_i\leq z_i^{\mathrm{th}}, i=1, 2 \dots L$$
The $z_i\leq z_i^{\mathrm{th}}$ conditions ensures a configuration doesn't include supercritical sites. The $0\leq z_i$ condition omits stable configurations that aren't reachable: as there is no way for a grain to be relaxed from a site with $z_i\leq 1$.

Now: if a configuration cannot be reached through relaxation that includes grains leaving the system (and possibly some steps after that relaxation), it must be transient: as the number of grains in it is bigger than the number of grains in any configuration that could eventually lead to this configuration. This means that for a system with length $L$, all configurations with $h_L \leq 0$ are transient, as the first grain at site $L$ cannot be relaxed under any circumstance. Now that we have this, we can interpret the sites $i=1, 2\dots L-1$ as a system of length $L-1$ and apply the same argument to determine that all configurations with $h_{L-1} \leq 1$ are transient, since once $h_L$ becomes $1$ and $h_{L-1}$ climbs to $2$, the value fo $h_{L-1}$ cannot drop below $2$, since there isn't any state with $h_{L-1} = 2$ for which the site at $L-1$ is supercritical. By induction, we can show that any configuration for which there exists $j$ such that $h_j\leq L-i$ is transient.

From this we see that the recurrent configuration with the smallest height of the pile is the configuration for which
$$h_i=L+1-i\hbox{ and therefore }z_i=1, i = 1, 2\dots, L$$
We can see this is recurrent, as it is possible to have $z_i^\mathrm{th}=1, i=1, 2\dots L$, which means that driving this configuration will iterate back to itself.

We can also see that the recurrent configuration with the largest height of the pile is equal to the possible configuration with the largest height of the pile, which is such that
$$h_i=2(L+1)-2i\hbox{ and therefore }z_i=2, i = 1, 2\dots, L$$
Indeed, having a larger pile would necessitate $\exists j:z_j>2$, which would render such configuration impossible.

Finally: once the system reaches the set of recurrent configurations, its height will belong to the set of heights of recurrent configurations, whose minimum is $h_{inf}(L)=L$ and maximum is $h_{sup}(L)=2L$. Therefore, after reaching the recurrent configurations, we will have
$$L\leq h(t; L) \leq 2L, \forall t > t_{recurrent}$$

This is exactly what we observe on the following plots: the height of the pile increases from its initial value of $0$ until it reaches the recurrent region, in which it randomly oscillates between the lower and upper bound of $L$ and $2L$ respectively:

INSERT PLOTS

\subsection{TASK 2b}
\subsection{TASK 2c}
We have
$$L \leq \langle h\rangle \leq 2L$$
in the steady state. Hence $\langle h\rangle$ scales with $L$

\textbf{Proof that $\langle t_c\rangle$ scales like $L^2$}

Let's investigate the step in which the first cross-over occurs. This step describes adding one grain to the configuration $A$ that directly precedes the first cross-over event. Resolving this driving step by relaxing all supercritical sites in $A$ (which will include the cross-over) shall result in a configuration $B$. Since no grains have left from the system up until $A$ is reached, the number of grains in $A$ is equal to the simulation time: $N(A)=t$. Therefore $t_c=N(A)+1$.

We shall have an ordered sequence of indices which describes the order in which relaxations of supercritical sites in $A$ were resolved - call it $R$. Relaxation of a site at $j, j>1$ must be preceded by a relaxation at site $j-1$ in the same step (as that is the only way to disturb the site at $j$ by adding a single grain and raising $z_j$ by $1$). Therefore, a cross-over event requires every site from $i=1$ to $i=L$ to be relaxed at least once during that step.

Now: consider the sub-sequence $M$ in $R$, which contains the last occurence of every index from $1$ to $L$, e.g.:
$$R=\mathbf{1}, 2, 3, \mathbf{2}, 3, 4, 5, \mathbf{4}, \mathbf{3}, \mathbf{5}, 6, 7\dots$$
(The elements of $M$ in this example are \textbf{bold}). We see that $M$ contains each index from $1$ to $L$ exactly once, but they're not necessarily in incremental order.

Now: if we switch up the order of relaxation on indices before the last occurence of the index $i$, then $h_i$ stays the same (by Abelianity). We know that in the normal order, site $i$ is supercritical at that point. Therefore, if we relax all the indices that are in $R$ before the last occurence of $i$, then $i$ will be supercritical.

Now: if site $i$ is supercritical, then relaxing any site other than $i$ will not change the fact that site $i$ is supercritical (since $z_i$ would have to decrease). Therefore we can actually relax any amount of indices occurring in $R$ \textit{after} the last occurrence of $i$: as long as all indices before the last ocurrence of $i$ have been relaxed, the site at $i$ will be supercritical.

Therefore, if we relax all the indices in $R$ that do \textbf{not} belong to $M$, we will be left with an altered configuration $A'$ such that if we start relaxing indices in $M$ in-order, the first non-relaxed index in $M$ will be always pointing at a supercritical site. This means that for any index $n$, we have four options:
\begin{enumerate}
\item If $n-1$ is before $n$ in $M$ and $n+1$ is after $n$ in $M$, then $z_n\geq 1$ in $A'$
\item If both $n-1$ and $n+1$ are before $n$ in $M$, then $z_n\geq 0$ in $A'$, but $z_{n-1}\geq 1$ in $A'$, otherwise the site at $n-1$ couldn't be supercritical when $n-1$ is reached in $M$.
\item If both $n-1$ and $n+1$ are after $n$ in $M$, then $z_n\geq 2$ in $A'$
\item If $n-1$ is after $n$ in $M$ and $n+1$ is before $n$ in $M$, then $z_n\geq 1$ in $A'$
\end{enumerate}

From this we see that $z_n\geq 0, n=1, 2\dots L$ in $A'$. By investigating every option, we see that $z_{n-1}+z_n\geq 1$ for any two neighbouring sites. This means that the number of grains in $A'$ has a lower bound of
\begin{eqnarray}
N(A')&\geq & \sum_{i=1}^{L}1+\left \lfloor{\frac{L-i}{2}}\right\rfloor\\
N(A')&\geq &\left(\left \lfloor{\frac{L}{2}}\right\rfloor +1\right)\frac{L}{2}
\end{eqnarray}
Since grain could only leave the site during the relaxation of $A$ (after the driving grain was inputted), we have $N(A')\leq N(A)+1$. Therefore

\begin{eqnarray}
N(A)&\geq &\left(\left \lfloor{\frac{L}{2}}\right\rfloor +1\right)\frac{L}{2}-1
\end{eqnarray}

Since $A$ has to be a possible configuration, it requires $z_i\leq 2, i=1, 2\dots L$, which puts an upper bound on $N(A)$:
$$N(A)\leq L(L+1)$$
Therefore $t_c$ is bounded like so:
$$\left(\left \lfloor{\frac{L}{2}}\right\rfloor +1\right)\frac{L}{2}\leq t_c \leq L(L+1)+1$$
Since both the lower and the upper bounds of $t_c$ scale like $L^2$ for all values of $L$, $t_c$ (and $\langle t_c\rangle$ as well) must also scale like $L^2$.

% Flushes all waiting graphics to appear now
%\clearpage

\subsection{References}

References probably not really needed for a project but you might have to cite something.  Follow the standard advice on plagiarism.  For \LaTeX\ issues and bibliographies see this file \texttt{{\jobname}.tex} for comments on how (not) to do it. For instance, these two publications are great \cite{KN05,EL09}.

% ********************************************************************
\newpage
\noindent
\emph{NB: This page (acknowledgement and bibliography) does not count towards the 2500 word limit nor the 16 page limit.}
\vspace*{2cm}

\section*{Acknowledgements}

Not required but you might want to thank A.Demonstrator for help.

% *****************************************************************
%
% BIBLIOGRAPHY
%
% **************************************************************
% 1) DO NOT WASTE TIME WITH BibTeX unless you REALLY KNOW WHAT YOU ARE DOING.
%    Just write it by hand.
% 2) NO bibliography is really needed for the projects
%    but you might cite something you read or which provided an image.
% 3) If we can get the item from the information you provided then it is fine.
%    The exact format is not important, but a consistent format is important.
%    Copy the style form one publication.
%    If space not an issue, titles of papers are useful.

\begin{thebibliography}{99}
\bibitem{E14}
  T.S.\ Evans, \emph{Slides for Networks course}, Physics Dept., Imperial College London, 2014, downloaded from Blackboard 2nd February 2015.
\bibitem{KN05}
  K.Christensen and N.Maloney,
  \emph{Complexity and Criticality},
  Imperial College Press, London, 2005.
\bibitem{EL09}
  T.S.\ Evans and R.\ Lambiotte,
  ``Line Graphs, Link Partitions and Overlapping Communities'',
  Phys.Rev.E.\ \textbf{80} (2009) 016202.
\end{thebibliography}


\end{document}
